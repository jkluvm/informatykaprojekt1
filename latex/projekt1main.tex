\documentclass[10pt,a4paper]{article}

\usepackage{gensymb}
\usepackage{hyperref}

\input{settings/packages}   % ścieżka względna do katalogu z ustawieniami
\graphicspath{{images/}}    % stała ścieżka względna do katalogu z  obrazkami.

\newcommand{\logoGIK}{settings/WGiK-znak.png}
\newcommand{\authorName}{Agata Wyrzykowska  \\ grupa 3, Numer Indeksu: 312109}

\newcommand{\titeReport}{Projekt 1, 12.04.2022r.}
\newcommand{\titleLecture}{Informatyka geodezyjna \\ sem. IV, projekt, rok akad. 2021-2022}
\newcommand{\kind}{report}
\newcommand{\mymail}{\href{mailto:01160139@pw.edu.pl}{01160139@pw.edu.pl}}
\newcommand{\supervisor}{....}
\newcommand{\gikweb}{\href{www.gik.pw.edu.pl}{www.gik.pw.edu.pl}}
\newcommand{\institut}{Zakład Geodezji Wyższej i Astronomii}
\newcommand{\faculty}{Wydział Geodezji i Kartografii}
\newcommand{\university}{Politechnika Warszawska}
\newcommand{\city}{Warszawa}
\newcommand{\thisyear}{2022}

\pdfinfo
{
	/Title       (GIK PW)
	/Creator     (TeX)
	/Author      (Agata Wyrzykowska)
}

% ------------------------- POCZATEK DOKUMENTU -------------------
\begin{document}
	% ----------------------------------------------------------------
	% ----------------------------  Title page
	% ----------------------------------------------------------------
	\begin{center} 
		\rule{\textwidth}{.5pt} \\
		\vspace{1.0cm} %odstep pionowy
		\includegraphics[width=.4\paperwidth]{\logoGIK}
		\vspace{0.5cm} \\
		\Large \textsc{\titeReport}
		\vspace{0.5cm} \\  
		\large \textsc{\titleLecture}
		\vspace{0.5cm}\\
		\textsc{\authorName}  \\
		\mymail \\
		\textsc{\faculty}, \textsc{\university}  \\ 
		\city, \today
	\end{center} 
	\rule{\textwidth}{1.5pt}
	
	
	% ---------------------------------------------------------------
	% ----------------------------  Table of content
	% ----------------------------------------------------------------
	\tableofcontents 								% wyświetla spis treści
	%\addcontentsline{to}{chapter}{Spis treści} 	% dodaje pozycję do spisu treści
	% \listoffigures  								% wyświetla spis rysunków
	%\addcontentsline{toc}{chapter}{Lista rysunków} % dodaje pozycję do spisu treści
	% \listoftables 									% wyświetla spis rysunków
	%\addcontentsline{toc}{chapter}{lista tabel}	% dodaje pozycję do spisu treści
	\newpage
	
	\section{Opis zadania}
	
	W celu wykonania pierwszego projektu należało przygotować aplikację która będzie potrafiła przeliczać dowolne współrzędne. Aby sprawdzić poprawność napisanego programu należało również wykonać napisać kod testujący wszystkie napisane w nim funkcje. Dodatkowo został napisany również pseudokod oraz schemat blokowy podanych w programie funkcji. \\
	Wszystkie pliki z programami oraz plik ze współrzędnymi należało zaimportować do GitHuba: \href{https://github.com/jkluvm/informatykaprojekt1}{\textbf{link do strony internetowej z kodem programu}}.
	
	\subsection{Instrukcja obsługi programu.}
	Całość aplikacji została napisana w klasie. Na samym początku należy skorzystać z funkcji inicjującej, która pozwoli nam zdefiniować rodzaj elipsoidy i na podstawie wybranych parametrów będzie można opierać swoje dalsze obliczenia. Do wyboru można mamy elipsoidę \textbf{GRS80} oraz \textbf{WGS84}. W przypadku wybrania innej dowolnej elipsoidy program zwróci informację, że dana elipsoida nie została zaimportowana do funkcji i należy skorzystać z którejś z powyższych. Po wyznaczeniu dużej i małej półosi, algorytm oblicza również spłaszczenie Ziemi oraz kwadrat mimośrodu.\\
	\\
	Następną funkcją w programie jest funkcja \textbf{dms}, która służy do zamiany stopni dziesiętnych na stopnie, minuty i sekundy z dokładnością do pięciu miejsc po przecinku. Nie jest ona niezbędna, służy tylko i wyłącznie stylistycznemu  przedstawieniu wyników w stopniach. \\
	\\
	Następną funkcją jest \textbf{hirvonen}, który jest algorytmem transformującym współrzędne ortokartezjańskie na geodezyjne z dokładnością do ok. 1 cm. Do jej uruchomienia algorytmu należy podać współrzędne X, Y, Z dowolnego punktu. Funkcja korzysta również z parametrów elipsoidy wyznaczanych przez funkcję inicjującą. \\
	\\
	Kolejna funkcja jest odwrotną do hirvonena - \textbf{blh2xyz}. Służy do przekształcania współrzędnych geodezyjnych na ortokartezjańskie. Wykorzystuje parametry elipsoidy wyznaczone w funkcji inicjującej. \\
	\\
	Funkcja \textbf{xyz2neu} służy do przekształcania współrzędnych ortokartezjańskich na topocentryczne. Aby ją zainicjować należy podać współrzędne dwóch dowolnych punktów. Wykorzystany jest w nim opisany wyżej algorytm hirvonena, który pozwoli nam wyznaczyć fi, lam, h wybranego z punktów, które sa konieczne do wyznaczenia macierzy. \\
	\\
	Następnym algorytmem jest funkcja \textbf{uklad1992}, która zamienia przestrzenne współrzędne geodezyjne fi, lam na współrzędne płaskie x, y. W tej części również są wykorzystywane parametry elipsoidy z funkcji inicjującej. \\
	\\
	Na tej samej zasadzie działa funkcja \textbf{uklad2000}, jest ona jednak nieco bardziej skomplikowana, gdyż posiada wiele warunków, w zależności od tego w której części kraju znajdą się odwzorowywane punkty geodezyjne. \\
	\\
	Ostatnią funkcją jest algorytm \textbf{katy odl}, który wyznacza kąt azymutu i elewacji oraz odległości 2D i 3D między punktami. Aby wywołać funkcję, należy podać współrzędne ortokartezjańskie dwóch dowolnych punktów na powierzchni Ziemi. \\
	\newpage
	
	\section{Schematy blokowe wybranych funkcji}
	\subsection{Funkcja inicjująca parametry elipsoidy}
	\begin{center}
		\newcommand{\model}{settings/model.png}
		\includegraphics[width=.8\paperwidth]{\model}
	\end{center}
	\newpage
	
	\subsection{Funkcja hirvonen}
	\begin{center}
		\newcommand{\hirvonen}{settings/hirvonen.png}
		\includegraphics[width=.8\paperwidth]{\hirvonen}
	\end{center}
	\newpage
	
	\subsection{Funkcja blh2xyz}
	\begin{center}
		\newcommand{\blhxyz}{settings/blh2xyz.png}
		\includegraphics[width=.8\paperwidth]{\blhxyz}
	\end{center}
	\newpage
	
	\subsection{Funkcja xyz2neu}
	\begin{center}
		\newcommand{\xyzneu}{settings/xyz2neu.png}
		\includegraphics[width=.8\paperwidth]{\xyzneu}
	\end{center}
	
	\subsection{Funkcja uklad1992}
	\begin{center}
		\newcommand{\uklad}{settings/uklad1992.png}
		\includegraphics[width=.7\paperwidth]{\uklad}
	\end{center}

	\subsection{Funkcja uklad2000}
	\begin{center}
		\newcommand{\ukladd}{settings/uklad2000.png}
		\includegraphics[width=.6\paperwidth]{\ukladd}
	\end{center}
	
	\subsection{Funkcja katy odl}
	\begin{center}
		\newcommand{\katy}{settings/katy.png}
		\includegraphics[width=.7\paperwidth]{\katy}
	\end{center}
	
	\newpage
	\section{Pseudokod}
	\rule{\textwidth}{.5pt} \\
	Wybierz jedną z funkcji: \\
	1. \textbf{init}: \\
	wyznacz model elipsoidy na bazie której będą wykonywane transformacje \\
	2. \textbf{dms} \\
	przekształć kąty dziesiętne na kąty wyrażone w stopniach, minutach i sekundach do pięciu miejsc po przecinku \\
	3. \textbf{hirvonen} \\
	wykonaj transformację ze współrzędnych geocentrycznych na geodezyjne \\
	4. \textbf{blh2xyz} \\
	wykonaj transformacje ze współrzędnych geodezyjnych na geocentryczne \\
	5. \textbf{xyz2neu} \\
	wykonaj transformacje ze współrzędnych geocentrycznych na współrzędne topograficzne \\
	6. \textbf{uklad1992} \\
	wykonaj transformacje ze współrzędnych geodezyjnych na współrzędne płaskie układu 1992 \\
	7. \textbf{uklad2000} \\
	wykonaj tranformacje ze współrzędnych geodezyjnych na współrzędne plaskie układu 2000 \\
	8. \textbf{katy odl} \\
	wykonaj algorytm na wyznaczenie kąta azymutu, kąta elewacji oraz odległości 2D i 3D \\
	\rule{\textwidth}{.5pt} \\
	\newpage
	
	\section{Historia commitów w pierwszym repozytorium.}
	Czas wykonania screenów: 14.04.2022r., godz. 00:10
	\begin{center}
		\newcommand{\commit}{settings/commits2.png}
		\includegraphics[width=.8\paperwidth]{\commit}
		\\
		\newcommand{\committ}{settings/commits1.png}
		\includegraphics[width=.8\paperwidth]{\committ}
	\end{center}
	
\end{document}